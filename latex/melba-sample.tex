\documentclass[twoside,11pt]{article}

% melba package. use options:
% - 'submission' for submission.
% - 'final' for MELBA accepted papers only
\usepackage[submission]{melba}




% Any additional packages needed should be included after jmlr2e.
% Note that jmlr2e.sty includes epsfig, amssymb, natbib and graphicx,
% and defines many common macros, such as 'proof' and 'example'.
%
% It also sets the bibliographystyle to plainnat; for more information on
% natbib citation styles, see the natbib documentation, a copy of which
% is archived at http://www.jmlr.org/format/natbib.pdf


% often used packages
\usepackage{amsmath,amsfonts}

% add packages here

% Definitions of handy macros can go here
\newcommand{\dataset}{{\cal D}}
\newcommand{\fracpartial}[2]{\frac{\partial #1}{\partial  #2}}


% Heading for publication (ignore during submission).
% arguments are:
% {article-id}{year}{pages}{date submitted}{date published}{author-full-names}{special-issue-title}{special-issue-editors}
%
% Publications instructions for authors of FINAL ACCEPT papers, before uploading final proof to scholastica:
%  article-id: This is provided upon acceptance from the Associate Editor
%  article-link: This is provided upon acceptance from the Associate Editor
%  year: the year of publication (the year you plan to upload to arXiv the final version)
%  pages: "1-X" where X is the number of pages in this final paper (including appendices)
%  date-submitted: Date submitted to MELBA: mm/yyyy
%  date-published: Today's date: mm/yyyy
%  author-full-names: Last names of all the authors, separated by commas or 'and' before final author.
%  special-issue-title: If part of a special issue, name of the issue *as given by the Associate Editor*. Otherwise, leave blank.
%  special-issue-editors: If part of a special issue, editors of the issue *as given by the Associate Editor*. Otherwise, leave blank.
\melbaheading{0}{https://www.melba-journal.org/article/XX-AA}{2020}{1-?}{mm/yyyy}{mm/yyyy}{Dalca and Sabuncu}{}{}

% Short headings should be {running head} and {authors last names}
\ShortHeadings{Autoencoder for Shapes of Brain Structures}{Dalca and Sabuncu}
\firstpageno{1}


% Title
% If the title spans several lines, authors could decide where the title should be split using \\
\title{An Inspiring Title \\ for the MELBA Journal Sample Article}


\author{\name Adrian V. Dalca \email adalca@mit.edu \\  % start right after \author{, or there will be an extra space
	\addr EECS, Massachusetts Institute of Technology, Cambridge, MA, USA
	\AND
	\name Mert R. Sabuncu \email msabuncu@cornell.edu \\
	\addr School of Electrical and Computer Engineering, Cornell University, Ithaca, NY, USA
}




\begin{document}

% top matter
\maketitle

% abstract
\begin{abstract}%   <- trailing '%' for backward compatibility of .sty file
We develop a learning framework for building deformable templates, which play a fundamental role in many image analysis and computational anatomy tasks. Conventional methods for template creation and image alignment to the template have undergone decades of rich technical development. In these frameworks, templates are constructed using an iterative process of template estimation and alignment, which is often computationally very expensive. Due in part to this shortcoming, most methods compute a single template for the entire population of images, or a few templates for  specific sub-groups of the data. In this work, we present a probabilistic model and efficient learning strategy that yields either universal or \textit{conditional} templates, jointly with a neural network that provides efficient alignment of the images to these templates. We demonstrate the usefulness of this method on a variety of domains, with a special focus on neuroimaging. This is particularly useful for clinical applications where a pre-existing template does not exist, or creating a new one with traditional methods can be prohibitively expensive.
%
Our code is available at~\url{http://yoururl.com}.
\end{abstract}

% keywords
\begin{keywords}
  Machine Learning, Image Registration
\end{keywords}



% Introduction (or first section)
\section{Introduction}

A deformable template is an image that can be geometrically deformed to match images in a dataset, providing a common reference frame. Templates are a powerful tool that enables the analysis of geometric variability. They have been used in computer vision, medical image analysis, graphics, and time series signals.


%%%%%%%%%%%%%%%%%%%%%%%%%%%%%%%%%%%%%%%%%%%%%%%%%%%%%%%%%%%%%%%%%%%%%%%%%%%
% Related works
%%%%%%%%%%%%%%%%%%%%%%%%%%%%%%%%%%%%%%%%%%%%%%%%%%%%%%%%%%%%%%%%%%%%%%%%%%%
% Make sure to put your work into context and include apporpriate citations.
% We do not have limits on citation counts.
\section{Related Works}

Spatial alignment, or registration, between two images is a building block for estimation of deformable templates. Alignment usually involves two steps: a global affine transformation, and a deformable transformation (as in many optical flow applications).

Use \verb|\cite{}| for reference that is part of the sentence, and \verb|\citep{}| for references in parenthesis. For example, \cite{viola1997alignment} is awesome. Also, this is a citation~\citep{viola1997alignment}.



% A methodological, model, or similar section often comes here.
\section{Methods}

\subsection{Equations}

We estimate the deformable template parameters~$\theta_t$ and the deformation fields for every data point using maximum likelihood. Letting~$\mathcal{V} = \{\boldsymbol{v}_i\}$ and~$\mathcal{A} = \{a_i\}$,
%
\begin{align}
\hat{\theta_t}, \hat{\mathcal{V}} &= \arg \max_{\theta_t, \mathcal{V}} \log p_{\theta_t}(\mathcal{V} | \mathcal{X},  \mathcal{A}) \nonumber \\
&= \arg \max_{\theta_t, \mathcal{V}} \log p_{\theta_t}(\mathcal{X} | \mathcal{V}; \mathcal{A}) + \log p(\mathcal{V}),
\label{eq:logpost}
\end{align}
%
where the first term captures the likelihood of the data and deformations, and the second term controls a prior over the deformation fields.

\begin{proof}
    Awesome proof.
\end{proof}



%%%%%%%%%%%%%%%%%%%%%%%%%%%%%%%%%%%%%%%%%%%%%%%%%%%%%%%%%%%%%%%%%%%%%%%
% Mandatory Sections. Please complete, especially for final publication
%%%%%%%%%%%%%%%%%%%%%%%%%%%%%%%%%%%%%%%%%%%%%%%%%%%%%%%%%%%%%%%%%%%%%%%

% Acknowledgements.
% Please include any funding, intellectual contributions not included in the authorship, and any other acknowledgements.
\acks{This work was supported by grants X, Y and Z. We also acknowledge important conversations with our colleagues A, B and C.}

% Ethical Standards.
% Please edit with the appropriate ethics considerations for your work. Include any pertinent IRB information, etc.
%
% Please note that the submission requirements included:
% The work presented must follow appropriate ethical standards in conducting research and writing the manuscript, following all applicable laws and regulations regarding treatment of animals or human subjects.
\ethics{The work follows appropriate ethical standards in conducting research and writing the manuscript, following all applicable laws and regulations regarding treatment of animals or human subjects.}

% Conflict of Interest
% Declaration of possible conflicts of interest: Authors must disclose any financial, organisational, commercial or personal conflicts of interest that might bias their work.
% If no conflicts, please say "We declare we don't have conflicts of interest."
\coi{The conflicts of interest have not been entered yet.}



% Manual newpage inserted to improve layout of sample file - not
% needed in general before appendices/bibliography.
% \newpage

\appendix % optional
\section*{Appendix A.}


In this appendix we prove the central theorem and present additional experimental results.
\noindent


{\noindent \em Remainder omitted in this sample. }


\vskip 0.2in
\bibliography{sample}

\end{document}
